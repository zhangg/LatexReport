\documentclass{beamer}

\setbeamertemplate{footline}[frame number]
\setbeamertemplate{navigation symbols}{}

\usetheme{Singapore}
\usecolortheme{seahorse}
%\usetheme{CambridgeUS}
%\usecolortheme{seahorse}
\usepackage{CJKutf8}
%\setCJKmainfont{SimSun}

\title{考核报告}
\author{张刚}
\institute{}

\begin{document}
\begin{CJK*}{UTF8}{}
\CJKfamily{gbsn} %中文支持

\maketitle

%\begin{frame}
%  \frametitle{}
%  \tableofcontents
%\end{frame}

\section{数据集测试与封装}
\begin{frame}
  \frametitle{数据集测试}
  \begin{itemize}
    \item BES-Test,基于DIRAC版本v6r10-pre8,功能都是可用的 
    \item BES-Production,基于DIRAC版本v6r10-pre15,发现问题并反馈。
    \item 升级BES-Production的DIRAC版本为v6r10-pre17之后,重新测试功能可用。 
  \end{itemize}
\end{frame}
\begin{frame}
  \frametitle{数据集功能封装}
  \begin{itemize}
    \item 对API进行了修改。
    \begin{itemize}
      \item 整理Badger.py中的API,删除旧的数据集功能。
      \item 将原有将每个新功能封装成函数写进Badger.py
    \end{itemize}
    \item 通过调用Badger.py中的相关函数封装成客户端命令供用户使用。
      \begin{itemize}
        \item besdirac-dms-dataset-add
        \item besdirac-dms-dataset-list
        \item besdirac-dms-dataset-filelist
        \item besdirac-dms-dataset-check 
        \item ...
      \end{itemize}
    \item 使用方法举例
      \begin{itemize}
        \item besdirac-dms-dataset-add 663\_qqbar\_stream001 /bes/File bossVer=663 eventType=qqbar streamID=stream001
        \item besdirac-dms-dataset-describe datasetname
          \begin{itemize}
            \item Status,Totalsize,NumberOfFiles,Owner,OwnerGroup
          \end{itemize}
      \end{itemize}
    \item 将dataset相关功能写进用户使用手册。
  \end{itemize}
\end{frame}
%\begin{frame}
%  \frametitle{使用方法}
%  \begin{itemize}
%    \item 
%  \end{itemize}
%\end{frame}
\section{优化传输工具}
\begin{frame}
  \frametitle{传输工具优化}
  \begin{itemize}
    \item 功能优化,提供不同参数选项。 
      \begin{itemize}
        \item -n :参数为datasetname,下载该数据集所包含的文件。 
          \begin{itemize}
            \item besdirac-dms-get-files -n name1 
          \end{itemize}
        \item -m:参数为一个查询条件,下载符合条件的文件。
          \begin{itemize}
            \item besdirac-dms-get-files -m bossVer=664 resonance=jpsi runL>12345 
          \end{itemize}
        \item -r:参数为DFC中的目录,下载该目录下所有文件。
          \begin{itemize}
            \item besdirac-dms-get-files -r /bes/File/psipp/664p01/mc/qqbar/round04/stream001 
          \end{itemize}
      \end{itemize}
  \end{itemize}
\end{frame}
\begin{frame}
  \frametitle{传输工具优化}
  \begin{itemize}
    \item 断点重传 
      \begin{itemize}
        \item 对每一个文件进行标记,初始标记为0,如果该文件下载成功,标记为1。 
        \item 如果因为意外导致程序死掉,重新运行后,会从第一个标记为0的文件下载,实现断点重传。
      \end{itemize}
    \item 文件校验 
      \begin{itemize}
        \item 对下载前后文件的大小尽心对比。
        \item 如果不一致,则重新下载;一致则表明下载成功。 
      \end{itemize}
    \item 将传输工具使用方法写到wiki,共用户参考。
  \end{itemize}
\end{frame}
\section{撰写小论文}
\begin{frame}
  \frametitle{撰写小论文}
  \begin{itemize}
    \item 对前一段所做工作进行总结,撰写小论文 
    \item 完成初稿,并已投到《计算机应用研究》。 
    \item 找工作,准备笔试面试相关资料。
  \end{itemize}
\end{frame}
\begin{frame}
 \begin{center}
  \LARGE 谢谢 
 \end{center}
\end{frame}
\end{CJK*}
\end{document}
